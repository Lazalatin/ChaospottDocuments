\documentclass{article}
\usepackage[utf8]{inputenc}
\usepackage[ngerman]{babel}
\usepackage{hyperref}
\usepackage[%
    a4paper,%
    bindingoffset=5mm,%
    left=15mm,%
    right=15mm,%
    top=15mm,%
    bottom=25mm,%
    footskip=6mm]{geometry}

\usepackage{titlesec}
\titleformat*{\section}{\large\bfseries}
\titleformat*{\subsection}{\bfseries}
\titleformat*{\subsubsection}{\scshape}

\setlength{\parindent}{0pt}


\title{Datenschutzerklärung gemäß DSGVO Art.~13\\
    für die Mitgliederverwaltung des foobar~e.V.}
\date{Stand 2018-05-25}

\begin{document}
\maketitle

Wir erheben auf dem Mitgliedsantrag persönliche Daten.
Gemäß Datenschutzgrundverordnung~(DSGVO) Artikel~13 geben wir dazu die
folgende Datenschutzerklärung ab:


\section*{(1a) Name und Anschrift des Verantwortlichen}

\subsection*{Verantwortlicher}

foobar e.V.\\
Sibyllastr. 9\\
45136 Essen\\
Deutschland\\[2ex]
E-Mail: info@die-foobar.de


\subsection*{Vertreter}

Der Vorstand:\\[2ex]
C. Roschow\\
S. Surminski\\
J. Stuber


\section*{(1b) Datenschutzbeauftragter}

Keiner, da weniger als 10 Personen mit der Verarbeitung
personenbezogener Daten beschäftigt sind (BDSG §4f Satz 4).


\section*{(1c)}

\subsection*{Zwecke der Verarbeitung}

Wir verarbeiten personenbezogene Daten unserer Mitglieder grundsätzlich
nur, soweit dies zur Vereinsführung erforderlich ist.
Regelmäßig sind dies z.B. die Einladung zu Versammlungen,
die Prüfung ob Mitgliedsbeiträge bezahlt wurden,
sowie die Ausstellung von Zuwendungsbescheinigungen.
%
Die Emailadresse kann bei Einwilligung des Mitglieds für die Einladung
zu Versammlungen verwendet werden.
%
Die Emailadresse wird zur sonstigen Kommunikation mit dem Mitglied
verwendet. Insbesondere wird sie zur internen Mailingliste
hinzugefügt, damit das Mitglied vereinsinterne Informationen erhält.
Gegebenenfalls ist es möglich, sich dort wieder abzumelden.


\subsection*{Rechtsgrundlage der Verarbeitung}

Rechtsgrundlage für die Verarbeitung ist DSGVO Art.~6~(1b), da die
Vereinsmitgliedschaft rechtlich gesehen ein Vertrag zwischen Mitglied
und Verein ist.


\section*{(1d) Berechtigte Interessen}

Entfällt, da die Daten nicht auf Grundlage von Art.~6~(1f)
"`berechtigte Interessen"' erhoben werden.


\section*{(1e) und (1f) Weitergabe}

Entfallen, da die erhobenen Daten nicht an Dritte weitergegeben werden.


\section*{(2a) Dauer der Speicherung}

Die Löschung oder Sperrung erfolgt in einem angemessenen Zeitraum nach
Ende der vorgeschriebenen Aufbewahrungsfristen.


\section*{(2b) Rechte}

Es bestehen Rechte auf Auskunft (siehe Art.~15),
Berichtigung (Art.~16),
Löschung (Art.~17),
Einschränkung der Verarbeitung von Daten (Art.~18)
sowie der Übertragung von Daten (Art.~20).


\section*{(2c) Widerruf der Einwilligung}

Entfällt, da die Daten nicht auf der Rechtsgrundlage einer
Einwilligung nach Art.~6~(1a) oder Art.~9~(2a) verarbeitet werden.


\section*{(2d) Beschwerderecht}

Es besteht das Recht, sich bei der Aufsichtsbehörde zu beschweren:

\medskip
Landesbeauftragte für Datenschutz und Informationsfreiheit Nordrhein-Westfalen\\
Helga Block\\
Postfach 20\,04\,44\\
40102 Düsseldorf\\ 
\url{https://www.ldi.nrw.de/}


\section*{(2e) Erforderliche Daten}

Name und Postanschrift sind erforderlich zur Identifikation, für die
Benachrichtigung des Mitglieds über Versammlungen und für sonstige
formale Vereinsangelegenheiten wie zum Beispiel Satzungsänderungen.
%
Die Informationen zum Beitragshöhe und -turnus sind erforderlich
für die Prüfung ob Mitgliedsbeiträge bezahlt wurden.
%
Die Emailadresse ist für die sonstige Kommunikation mit dem Mitglied
erforderlich.
%
Die GPG-Schlüssel-Id ist nicht erforderlich.


\section*{(2f) Automatisierte Entscheidungsfindung}

Es findet keine automatisierte Entscheidungsfindung statt.

\end{document}
