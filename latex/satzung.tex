\documentclass[10pt,a4paper]{scrartcl}
\usepackage[utf8]{inputenc}
\usepackage[T1]{fontenc}
\usepackage[ngerman]{babel}
\usepackage[hidelinks]{hyperref}
\usepackage[top=30mm, bottom=45mm, left=35mm, right=35mm]{geometry}
\usepackage{enumitem}
\renewcommand{\familydefault}{\sfdefault}
\renewcommand{\thesection}{§\arabic{section}}

\begin{document}
\noindent
{\huge\bfseries Satzung des foobar e. V.}\\[0.5cm]
Errichtet auf der Gründungsversammlung am 15. September 2010\\
Geändert auf der Mitgliederversammlung am 02. März 2011\\
Geändert auf der Mitgliederversammlung am 06. Dezember 2015


\section*{Präambel}

Die Wandlung von der Industrie- zur Dienstleistungs- und
Informationsgesellschaft unserer Zeit wäre ohne die Erfindung des
Computers nicht denkbar gewesen. Die automatisierte Datenverarbeitung
und -übertragung birgt weitreichende Möglichkeiten, aber auch Risiken,
für die Gesellschaft und den Einzelnen. Informations- und
Kommunikationstechnologie verändert das Verhältnis des Menschen zur
Maschine und der Menschen zueinander, während das technische
Verständnis in Gesellschaft und Gesetzgebung dem Fortschritt oft weit
hinterherhinkt.  Konflikte mit Grundrechten wie beispielsweise dem
Recht auf Privatsphäre oder dem Recht auf informationelle
Selbstbestimmung sind daher vorprogrammiert.  Obsolete
Geschäftsmodelle werden zum Nachteil der Allgemeinheit künstlich am
Leben erhalten. Die foobar ist eine Gemeinschaft von Menschen, deren
Mitglieder sich grenzüberschreitend und unabhängig von Alter,
Geschlecht, Herkunft und gesellschaftlicher Stellung für die
Informationsfreiheit einsetzen. Sie beschäftigt sich zudem kritisch
und differenziert mit den Auswirkungen von Technologie auf die
Gesellschaft und das Individuum und fördert die Aufklärung der
Gesellschaft in diesem Bereich.  Sie verschreibt sich der Erlangung,
dem Austausch und der Demokratisierung von Wissen und fördert die
Entwicklung zur Informationsgesellschaft. Dazu fordert sie ein neues
Menschenrecht auf weltweite, ungehinderte Kommunikation im Sinne der
freien Meinungsäußerung und des Austausches zwischen allen
Menschen. Ihre Mitglieder wollen verstehen und das eigene Wissen
mehren, um es mit anderen zu teilen. Sie analysieren kritisch neue
Technologien und damit zusammenhängende, gesellschaftliche Phänomene
und politische Entscheidungen und versuchen, Probleme und Gefahren für
Gesellschaft und Demokratie zu erkennen und die Allgemeinheit auf
diese aufmerksam zu machen.  Die foobar informiert alle interessierten
Bürger darüber, wie diese ihr Recht auf Privatsphäre und
informationelle Selbstbestimmung besser wahrnehmen und schützen
können. Sie fühlt sich der Wahrung von Demokratie, Freiheit und
Toleranz verpflichtet. Rassismus, Sexismus und andere Formen der
Diskriminierung duldet sie nicht.


\clearpage
\section{Name, Sitz, Geschäftsjahr}

\begin{enumerate}[label=(\arabic*)]
\item
  Der Verein führt den Namen foobar.\\
  Der Verein wird in das Vereinsregister eingetragen und dann um den Zusatz "`e.V."' ergänzt.
\item
  Der Verein hat seinen Sitz in Essen.
\item
  Das Geschäftsjahr ist das Kalenderjahr.
\end{enumerate}


\section{Zweck und Gemeinnützigkeit}

\begin{enumerate}[label=(\arabic*)]
\item
  Der Verein dient der Förderung der Erziehung, Volks- und Berufsbildung einschließlich
  der Studentenhilfe.
  Der Vereinszweck soll unter anderem durch folgende Mittel erreicht werden:
  \begin{itemize}
  \item
    Regelmäßige öffentliche Vorträge, Workshops, Diskussions- und
    Informationsveranstaltungen zu Themen aus Wissenschaft, Kultur, Technik,
    Gesellschaft und Politik,
  \item
    Öffentlichkeitsarbeit im Sinne der Aufklärungsfunktion und zur Information
    über Veranstaltungen des Vereins in allen dazu geeigneten Medien,
  \item
    Förderung des schöpferisch-kritischen Umgangs mit Technologie durch
    Arbeits- und Erfahrungsaustauschkreise,
  \item
    Zusammenarbeit und Austausch mit nationalen und internationalen
    Gruppierungen, deren Ziele mit denen des Vereins vereinbar sind,
  \item
    Zusammenarbeit mit Schulen und anderen Bildungseinrichtungen, um den
    kompetenten und kritischen Umgang mit digitalen Medien im Hinblick auf die
    informationelle Selbstbestimmung zu fördern,
  \item
    Durchführung von Veranstaltungen und Projekten zur Förderung
    bürgerkontrollierter Infrastruktur und quelloffener Soft- und Hardware,
  \item
    Durchführung von Veranstaltungen und Projekten zur Förderung von
    Demokratie, Toleranz, Freiheit und Völkerverständigung.
  \end{itemize}
\item
  Der Verein verfolgt ausschließlich und unmittelbar gemeinnützige Zwecke im Sinne des
  Abschnitts „Steuerbegünstigte Zwecke“ der Abgabenordnung.
  Er ist selbstlos tätig und verfolgt nicht in erster Linie eigenwirtschaftliche Zwecke.
  Die Mitglieder erhalten keine Zuwendungen aus Mitteln des Vereins.
  Mittel des Vereins dürfen nur für die satzungsmäßigen Zwecke verwendet werden.
  Es darf keine Person durch Ausgaben, die dem Zwecke des Vereins fremd sind, oder
  durch unverhältnismäßig hohe Vergütungen begünstigt werden.
\end{enumerate}


\section{Erwerb und Verlust der Mitgliedschaft}

\begin{enumerate}[label=(\arabic*)]
\item
  Erwerb der Mitgliedschaft:
  \begin{enumerate}[label=(\alph*)]
  \item
    Natürliche Personen können ordentliche Mitglieder des Vereins werden.
  \item
    Natürliche und juristische Personen können Fördermitglieder des Vereins werden.\\
    Fördermitglieder haben kein Stimmrecht in der Mitgliederversammlung.
  \end{enumerate}
\item
  Die Mitgliedschaft ist schriftlich, unter Anerkennung der Vereinssatzung zu beantragen.
  Über die Annahme entscheidet der Vorstand.\\
  Bei Aufnahmeanträgen Minderjähriger ist die schriftliche Zustimmung des gesetzlichen
  Vertreters erforderlich.
\item
  Die Mitgliederversammlung kann solche Personen, die sich besondere Verdienste um
  den Verein oder um die von ihm verfolgten satzungsgemäßen Zwecke erworben
  haben, zu Ehrenmitgliedern ernennen.\\
  Ehrenmitglieder haben alle Rechte eines ordentlichen Mitglieds.
  Sie sind von Beitragsleistungen befreit.
\item
  Die Mitgliedschaft erlischt durch Austritt, Ausschluss oder den Tod natürlicher
  Personen, bzw. Auflösung juristischer Personen.
\item
  Der Austritt wird durch Willenserklärung in Schriftform gegenüber dem Vorstand
  vollzogen. Die Beitragspflicht für den vom Mitglied gewählten Abrechnungszeitraum
  bleibt hiervon unberührt. Bereits gezahlte Beiträge werden nicht erstattet.
\end{enumerate}


\section{Rechte und Pflichten der Mitglieder}

\begin{enumerate}[label=(\arabic*)]
\item
  Die ordentlichen Mitglieder sind berechtigt, die Leistungen des Vereins
  in Anspruch zu nehmen.
\item
  Die Mitglieder sind verpflichtet, die festgesetzten Beiträge zu zahlen.
  Ist ein Mitglied mit seinem Beitrag mehr als 30 Tage im Verzug, so ruht seine
  Mitgliedschaft bis zum Ausgleich der Verbindlichkeiten.
\end{enumerate}


\section{Ordnungsmaßnahmen gegen Mitglieder}

\begin{enumerate}[label=(\arabic*)]
\item
  Ein Mitglied kann durch Beschluss des Vorstandes aus dem Verein ausgeschlossen
  werden, wenn es das Ansehen des Vereins schädigt, seinen Beitragspflichten
  nachhaltig nicht nachkommt, oder wenn ein sonstiger wichtiger Grund vorliegt.\\
  Der Vorstand muss dem auszuschließenden Mitglied den Beschluss in schriftlicher
  Form unter Angabe von Gründen mitteilen und ihm auf Verlangen eine Anhörung
  gewähren.
\item
  Gegen den Beschluss des Vorstands ist innerhalb einer Frist von zwei Monaten nach
  Zugang des Beschlusses die Anrufung der Mitgliederversammlung zulässig. Wird die
  Mitgliederversammlung angerufen, so hat der Vorstand, nach Maßgabe der Satzung,
  eine außerordentliche Mitgliederversammlung zeitnah einzuberufen. Bis zum
  Beschluss der Mitgliederversammlung ruht die Mitgliedschaft.
\end{enumerate}


\section{Beitrag}

\begin{enumerate}[label=(\arabic*)]
\item
  Der Verein erhebt einen regelmäßigen Mitgliedsbeitrag.
\item
  Art und Höhe des Mitgliedsbeitrags werden von der Mitgliederversammlung in einer
  Beitragsordnung beschlossen, welche nicht Teil dieser Satzung ist.
\item
  Im Einzelfall kann der Vorstand für ein Mitglied, auf dessen Antrag hin, einen von der
  Beitragsordnung abweichenden Betrag festsetzen.
\end{enumerate}


\section{Organe des Vereins}

Organe des Vereins sind
\begin{itemize}
\item
  die Mitgliederversammlung und
\item
  der Vorstand.
\end{itemize}


\section{Mitgliederversammlung}

\begin{enumerate}[label=(\arabic*)]
\item
  Oberstes Beschlussorgan ist die Mitgliederversammlung. Ihrer Beschlussfassung
  unterliegen alle in dieser Satzung oder geltendem Recht vorgesehenen Gegenstände,
  insbesondere
  \begin{itemize}
  \item
    die Wahl und Bestellung der einzelnen Vorstandsmitglieder,
  \item
    die Bestellung von Finanzprüfern,
  \item
    die Genehmigung des Finanzberichtes,
  \item
    die Entlastung des Vorstandes,
  \item
    die Änderung der Satzung
  \item
    die Verabschiedung der Beitragsordnung,
  \item
    die Genehmigung der Geschäftsordnung des Vorstandes,
  \item
    die Richtlinie über die Erstattung von Reisekosten und Auslagen,
  \item
    Anträge des Vorstandes und der Mitglieder,
  \item
    die Ernennung von Ehrenmitgliedern sowie
  \item
    die Auflösung des Vereins und die Beschlussfassung über die eventuelle
  \item
    Fortsetzung des aufgelösten Vereins.
  \end{itemize}
\item
  Die ordentliche Mitgliederversammlung findet einmal jährlich statt. Eine
  außerordentliche Mitgliederversammlung ist einzuberufen, wenn die Interessen des
  Vereins dies erfordern oder wenn ein Zehntel der Mitglieder dies schriftlich unter
  Angabe der Gründe beim Vorstand beantragt. Der Vorstand hat dann innerhalb einer
  Frist von zwei Wochen die Mitgliederversammlung einzuberufen.
\item
  Die Mitgliederversammlung ist durch den Vorstand schriftlich unter Einhaltung einer
  Ladungsfrist von zwei Wochen einzuberufen. Hierbei sind die Tagesordnung bekannt
  zu geben und die nötigen Informationen zugänglich zu machen. Die Einberufung der
  Mitgliederversammlung kann auch auf elektronischem Wege erfolgen, wenn das
  jeweilige Mitglied diesem ausdrücklich zugestimmt hat. Zur Wahrung der Frist reicht
  die Aufgabe der Einladung zur Post an die letzte bekannte Anschrift oder die
  Versendung an die zuletzt bekannte E-Mail-Adresse. Anträge zur Tagesordnung sind
  spätestens drei Tage vor der Mitgliederversammlung beim Vorstand in Schriftform
  einzureichen. Über die Behandlung von Initiativanträgen entscheidet die
  Mitgliederversammlung.
\item
  Die Mitgliederversammlung ist beschlussfähig, wenn mindestens zehn Prozent aller
  ordentlichen Mitglieder anwesend sind. Ist die Mitgliederversammlung aufgrund
  mangelnder Teilnehmerzahl nicht beschlussfähig, ist die darauf folgende
  ordnungsgemäß einberufene Mitgliederversammlung ungeachtet der Teilnehmerzahl
  beschlussfähig.
\item
  Jedes auf der Mitgliederversammlung anwesende ordentliche Mitglied, dessen
  Mitgliedschaft nicht ruht, hat eine Stimme. Beschlüsse über Satzungsänderungen
  zur Änderung des Vereinszwecks und über die Auflösung des Vereins bedürfen zu
  ihrer Wirksamkeit eine Zweidrittelmehrheit der Stimmen. In allen anderen Fällen
  genügt die einfache Mehrheit der abgegebenen Stimmen.
\item
  Beschlüsse über Satzungsänderungen und über die Auflösung des Vereins können nur
  in einer Mitgliederversammlung gefasst werden, zu der unter Angabe dieser Anträge
  im Wortlaut eingeladen wurde.
\item
  Die Mitgliederversammlung wählt aus ihrer Mitte einen Versammlungsleiter und einen
  Protokollführer. Über die Beschlüsse der Mitgliederversammlung ist ein Protokoll
  anzufertigen, das von dem Versammlungsleiter und dem Protokollführer zu
  unterzeichnen ist. Das Protokoll ist allen Mitgliedern zeitnah zugänglich zu machen.
\item
  Auf Antrag eines Mitglieds ist geheim abzustimmen. Personenwahlen haben immer
  geheim zu erfolgen.
\item
  Die Mitgliederversammlung ist öffentlich. Auf Beschluss der Mitgliederversammlung
  können Gäste zeitweise ausgeschlossen werden.
\end{enumerate}


\section{Vorstand}

\begin{enumerate}[label=(\arabic*)]
\item
  Der Vorstand besteht aus drei Mitgliedern:
  \begin{enumerate}[label=(\alph*)]
  \item
    dem Vorsitzenden,
  \item
    dem stellvertretenden Vorsitzenden und
  \item
    dem Schatzmeister.
  \end{enumerate}
\item
  Vorstand im Sinne des §26 BGB ist jedes Vorstandsmitglied. Die Vorstandsmitglieder
  sind generell jeweils alleinvertretungsberechtigt. Davon ausgenommen sind folgende
  Vorgänge, bei welchen der Verein durch mindestens zwei Vorstandsmitglieder
  vertreten wird:
  \begin{enumerate}[label=(\alph*)]
  \item
    Einstellung und Entlassung von Angestellten,
  \item
    gerichtliche Vertretung,
  \item
    Aufnahme von Krediten,
  \item
    Gründung, Erwerb und Veräußerung von Gesellschaften und Geschäftsanteilen von
    Gesellschaften zur Verwirklichung der satzungsgemäßen Ziele,
  \end{enumerate}
  Der Vorstand ist von den Beschränkungen des §181 BGB freigestellt.
\item
  (gestrichen)
\item
  Die Amtsdauer des Vorstands beträgt regelmäßig ein Jahr. Die Wiederwahl ist zulässig.
  Die Vorstandsmitglieder bleiben bis zur Wahl eines neuen Vorstands im Amt.
\item
  Ist mindestens ein Vorstandsmitglied dauerhaft an der Ausübung seines Amtes
  gehindert, so sind unverzüglich Nachwahlen anzuberaumen.
\item
  Der Schatzmeister überwacht die Haushaltsführung und verwaltet das Vermögen des
  Vereins. Er hat auf eine sparsame und wirtschaftliche Haushaltsführung hinzuwirken.
  Mit dem Ablauf des Geschäftsjahres stellt er unverzüglich die Abrechnung sowie die
  Vermögensübersicht und sonstige Unterlagen von wirtschaftlichem Belang den
  Finanzprüfern des Vereins zur Prüfung zur Verfügung.
\item
  Die Vorstandsmitglieder sind grundsätzlich ehrenamtlich tätig.
\item
  Der Vorstand gibt sich eine Geschäftsordnung, die der Genehmigung durch die
  Mitgliederversammlung bedarf.
\end{enumerate}


\section{Finanzprüfer}

\begin{enumerate}[label=(\arabic*)]
\item
  Zur Kontrolle der Haushaltsführung kann die Mitgliederversammlung Finanzprüfer
  bestellen. Nach Durchführung ihrer Prüfung informieren sie den Vorstand von ihrem
  Prüfungsergebnis und erstatten der Mitgliederversammlung Bericht.
\item
  Die Finanzprüfer dürfen dem Vorstand nicht angehören.\\
  Sie müssen nicht Mitglieder des Vereins sein.
\item
  Die Finanzprüfer sind grundsätzlich ehrenamtlich tätig.
\end{enumerate}


\section{Auflösung des Vereins}

Bei Auflösung des Vereins oder bei Wegfall steuerbegünstigter Zwecke fällt das
Vermögen des Vereins an den Verein zur Förderung des öffentlichen bewegten und
unbewegten Datenverkehrs e. V., beim Amtsgericht Bielefeld unter der
Registernummer HR 20 VR 2479 eingetragen. Sollte dieser Verein zum Zeitpunkt der
Auflösung nicht mehr bestehen, oder die Mitgliederversammlung etwas anderes
bestimmen, so fällt das Vermögen an eine von der Mitgliederversammlung zu
bestimmende Körperschaft des öffentlichen Rechts oder eine andere
steuerbegünstigte Körperschaft, die es unmittelbar und ausschließlich für
gemeinnützige Zwecke zu verwenden hat.

\end{document}
